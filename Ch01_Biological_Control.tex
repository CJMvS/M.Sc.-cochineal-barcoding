\section{Biological control}
\subsection{Overview}
\citet{VanDriesche2009} defines biological control as \textit{``The use of populations of natural enemies to suppress pest populations to lower densities, either permanently or temporarily''}. Biological control is a broad term encompassing the following methods: 
\vspace{0.4cm}
\begin{enumerate}
    \item `Classical', which is the most common approach, and refers to the use of natural enemies from the native range of a target pest (which share an evolutionary history) to achieve control.
    \item `Neo-classical' or `new association' methods, which makes use of non-native agents to control native species (which share no previous evolutionary history).
    \item `Augmentative', where the natural enemies of a target pest are mass-reared and released in order to augment naturally occurring enemy populations.
    \item `Conservation', which focuses on the maintenance of natural enemy populations in order to optimise their efficacy as parasitoids, pathogens, or predators. \\
    \citep{mcfadyen1998biological, collier2004critical,  VanDriesche2009, gullan2014insects, begg2017functional}.
\end{enumerate}
\vspace{0.4cm}

 \noindent Classical biological control is based primarily upon the enemy release hypothesis (ERH), which proposes that when a plant species is introduced into a new environment, it is released from its natural enemies (particularly specialist enemies) and is more readily able to spread quickly and invest resources into growth and reproduction \citep{keane2002exotic, mcevoy2002insect, liu2006testing}. This often involves energy trade-offs where chemical and structural defenses are down-regulated in favour of rapid growth and seed production \citep{blossey1995evolution, van1996optimal}. The ERH also predicts that it will be rare for the natural enemies of native plants to switch hosts and begin feeding on the invader, and that native generalist enemies will be more likely to feed on native species \citep{keane2002exotic}. \\
The benefits of biological control include: 
\vspace{0.4cm}
\begin{enumerate}
    \item Permanency of control and the self-perpetuation of agent populations.
    \item Host-specificity with little chance of non-target effects.
    \item High cost-to-benefit ratios relative to mechanical and chemical control methods.
    \item Being environmentally friendly (involving no toxic residues, and requiring a low energy input).
    \item Being able to naturally spread across the range of the host plant.
    \item The ability to be used in combination with other control methods as part of an integrated pest management (IPM) programme. \\
    \citep{hokannen1995, vanWilgen2004TheAfrica, culliney2005benefits, VanDriesche2009, zachariades2017assessing}.
\end{enumerate}

\subsection{The biological control pipeline}

A review by \citet{mcfadyen1998biological} summarised the steps involved in a biological control programme as follows: exploration in the weed's region/s of origin $\rightarrow$ selection of potential agents $\rightarrow$ host specificity testing $\rightarrow$ mass rearing of the chosen agent/s $\rightarrow$ releases into target areas $\rightarrow$ post-release evaluation and the implementation of integrated control strategies. The two central steps are that of agent selection and host specificity testing, which will briefly be reviewed here.

\subsubsection{Agent selection and prioritisation}

There is often an array of phytophagous species associated with a particular host plant \citep{frost1954numerical, southwood1961number}, but they do not all inflict the same mode and/or level of damage. \citet{wilson1960}, for example, reported that only five of the 52 insects imported for the control of \textit{Opuntia ficus-indica} in Australia could serve as damaging agents. The costs and time required for testing can quickly become unfeasible and impractical \citep{harris1991classical, mcfadyen1998biological, paterson2014prioritisation}. Host specificity testing is estimated to take approximately three scientist years and AUD\$460 000 (at 1997 prices) per agent \citep{mcfadyen1998biological}. The prioritisation of agents is thus integral to a biological control programme, such that only the most effective candidates are selected to deliver the highest level of targeted damage possible. This concept is central to this thesis, as the identification of the correct \textit{Dactylopius} lineage is necessary to inflict the most damage to the target cactus species.  Quoting \citet{mcfadyen1998biological}, predicting the success of an agent prior to release and selecting the most effective one is the `holy grail of weed biocontrol'. A number of factors need to be considered in order to select the most suitable candidate, including the following from a general scoring system originally proposed by \citet{harris1973selection} and modified by \citet{goeden1983}:
\vspace{0.4cm}
\begin{enumerate}
    \item The degree of host-specificity displayed (the agent must express suitable levels of specialisation for release in the introduced range).
    \item The level of direct and indirect damage caused to the host plant (where damage to vascular and mechanical support tissue is the most effective).
    \item Phenology of attack (where a prolonged attack is optimal that extends over the growing and/or reproductive phase of the target weed)
    \item The number of generations and progeny per generation produced (where a multivoltine and highly fecund agent is desired).
    \item Extrinsic mortality factors (where ideally only specialised natural enemies are associated with the agent, and competition by other species for the host plant is negligible).
    \item Feeding behaviour (where gregarious feeding is desired for greater damage).
    \item Compatibility with other agents.
    \item Distribution (where ideally the agent covers a large part of the range of the target weed).
    \item Previous successes as a biological control agent in other parts of the world.
\end{enumerate} 
\vspace{0.4cm}
Although the proposed scoring system was criticised for not being thorough enough \citep{wapshere1985effectiveness, cullen1995, kriticos2003}, it presented an initial framework and opened further discussion. \citet{cullen1995}, \citet{kriticos2003}, and \citet{klinken2006scientific} suggest that a more integrative approach to agent selection should be based on ecological modelling. This involves understanding the interactions between the agent and its host plant, and being able to predict how various factors such as disturbance regimes, competition, plant succession, different life stages of the host plant, and climatic variation may influence their dynamics \citep{mcevoy1999biological}. 
Closely linked to the concept of ecological modelling is that of climatic and genotypic matching; where the climatic compatibility between the native and introduced range is assessed, and the genetic variation (i.e. the number of haplotypes present) in both the host plant and agent is considered \citep{iline2004allozyme, dhileepan2006systematic, goolsby2006matching, robertson2008climate, Paterson2009UsingControl, VanDriesche2009, paterson2014prioritisation, fischbein2019modelling, jourdan2019sourcing, sutton2019searching}.
 Despite all the tools available, extrapolating results from a controlled environment to field conditions remains a major challenge (see for example \citet{buccellato2019post}). 

\subsubsection{Host specificity testing}

The safety of a biological control programme is of vital importance, and requires a thorough risk assessment of the potential non-target effects that could arise following the release of an agent into a target area \citep{mcevoy1996host, secord1996perils, mcfadyen1998biological, heard1999concepts, vanklinken1999host, schaffner2001host, heard2002host, VanDriesche2009}. This form of risk assessment is referred to as `host specificity testing', and aims to elucidate what the agent's host range breadth is, and its relative acceptability of different host plant species \citep{vanklinken1999host}.  \citet{vanklinken1999host} outline three central steps in the testing process, namely: 
\vspace{0.4cm}

\begin{enumerate}
    \item Identifying which stages in an agent's life history need to be host-specific
(larval instars might display differential feeding behaviours, and adults might damage non-target plants through feeding, oviposition, or the transferal of pathogens).
\item Estimating the fundamental host range, which is the range of host plants on which the agent is capable of completing its life cycle \citep{schaffner2001host}. This is tested through the use of the centrifugal-phylogenetic method proposed by \citet{wapshere1974strategy}. The method entails the testing of an array of plant species in the agent's host range; beginning with those most closely related to the target weed, working outwards to more distantly related species. Plants occurring in the same habitat, that share similar chemical and/or morphological traits, or serve as hosts to congeneric insects to that of the target weed are included in the testing procedure. Importantly, economically important crops and rare and endangered species in the same family as the target weed are prioritised. The testing procedure encompasses choice and no-choice tests \citep{schaffner2001host}, continuation, oogenesis, and open-field tests \citep{VanDriesche2009, Schaffner2018}.
\item Extrapolating to the field, which involves predicting what the realised host range will be under field conditions. The realised host range comprises the host plants that the agent can develop on under field conditions, and is initially predicted by taking surveys of related host plants occurring sympatrically with the target weed \citep{schaffner2001host}, followed by open field tests if possible \citep{balciunas1996comparison}.
\end{enumerate}
\vspace{0.4cm}
In cases where test results indicate that an agent will cause damage to one or more non-target species, a cost to benefit assessment is required to determine whether the relative value of the non-target species is higher than the damage caused by the target weed \citep{mcfadyen1998biological, VanDriesche2009}. Cost to benefit analyses have been used in the past to determine whether an agent should be released \citep{mcfadyen1990leaf, olckers1995interpreting, cristofaro1998biology, haye2006controlling} \\
Risk assessments are important to determine potential non-target effects associated with a released agent \citep{paynter2015relative, paynter2018making, hinz2019safe}. This is done by quantifying the performance of an agent on target and non-target plants, and calculating risk scores that can predict the probability of attack \citep{paynter2015relative}.

\subsubsection{Biological control track record}

Using the data compiled by \citet{Winston2014BiologicalWeeds.}, \citet{schwarzlander2018biological} summarised that up to the year 2012, 468 biological control agent species have been released onto 175 weed species (comprising 48 plant families) across the world. Of the agents released, 70.9\% established successfully. Post-release surveys determined that only 7\% of releases resulted in no impact at all on the target weed, and that 55\% of agents caused heavy, medium, or variable impact in at least one release. Of the targeted 175 weed species, 115 (65.7\%) have been controlled to some degree by at least one biological control agent. \\  
A review by \citet{hinz2019safe} stated that of the 457 species of biological control agents intentionally released between the years 1863 and 2008, 60 (13.1\%) were associated with non-target attacks (NTAs). Of those 60 NTAs, 42 involved attacks on native species. In agreement with \citet{sheppard2003global} and \citet{suckling2014magnitude}, the review concluded that less than 1\% of all weed biological control releases have the potential to result in negative non-target effects. Additionally, it found that the proportion of agent releases associated with NTAs has decreased despite the increase in releases over the years. This is most likely due to a combination of more stringent host specificity testing measures and stricter regulations regarding the release of biocontrol agents. \citet{hinz2019safe} emphasise that post-release monitoring should be as important as pre-release host specificity testing in order to accurately quantify agent impact and NTAs.