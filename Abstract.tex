\chapter*{Abstract}

The Cactaceae family comprises 15 genera and nearly 2000 species. With one exception, these are all native to the Americas. Numerous cactaceous species are invasive in other parts of the world, resulting in considerable damage to ecosystem functioning and agricultural practices. The most successful biological control agents used to combat invasive Cactaceae belong to the \textit{Dactylopius} genus (Hemiptera: Dactylopiidae), comprising eleven species. 
The Dactylopiidae are exclusively cactophagous and are usually host-specific. Some  intraspecific lineages of dactylopiids, often referred to as `biotypes', also display host-specificity, and are used to control particular species of invasive Cactaceae. To date, two lineages within \textit{Dactylopius opuntiae} (`ficus' and `stricta'), and two within \textit{D. tomentosus} (`cholla' and `imbricata') have been released in South Africa to control \textit{Opuntia ficus-indica} and \textit{O. stricta}, and \textit{Cylindropuntia fulgida} and \textit{C. imbricata}, respectively. The `californica var. parkeri' lineage is currently under consideration for release in South Africa for the control of \textit{C. pallida}. Australia has already released these five lineages, and approved the release of an additional three in 2017; namely \textit{D. tomentosus} `bigelovii', `cylindropuntia sp.', and `acanthocarpa x echinocarpa'. 
Many of the \textit{Dactylopius} species are so morphologically similar, and in the case of lineages, identical, that numerous misidentifications have been made in the past. These errors have had serious implications, such as failed attempts at the biological control of cactus weeds. 
% It is even suggested that the damaging \textit{D. opuntiae} `ficus' lineage entered Brazil because it was erroneously identified as being the domesticated \textit{D. coccus} species used for red dye production; and is now causing millions of dollars in damage to farms cultivating \textit{Opuntia ficus-indica}. 
This thesis aimed to generate a multi-locus genetic database to enable the identification of the species and lineages in the Dactylopiidae family, and to test its accuracy. Seven species were included in the analysis, including two lineages within \textit{D. opuntiae} and six within \textit{D. tomentosus}. Genetic characterisation was achieved through the DNA sequencing of three gene regions; namely mitochondrial 12S rRNA and cytochrome c oxidase I (COI), nuclear 18S rRNA, and fragment analysis using two inter-simple sequence repeats (ISSRs). 
Nucleotide sequences were very effective for species-level identification, where the 12S, 18S, and COI regions showed 100\%, 94.59\%, and 100\% identification accuracy rates, respectively. Additionally, the 12S and COI markers distinguished between half of the \textit{D. tomentosus} lineages (`californica', `cholla', and `imbricata'), with identification accuracies of 100\%. The `echinocarpa x acanthocarpa', `bigelovii', and `cylindropuntia sp.' lineages formed one clade. 
None of the DNA genetic markers showed a separation between the `ficus' and `stricta' lineages within \textit{D. opuntiae}. Fragment analysis through the use of ISSRs provided higher-resolution results, and addressed this gap by showing a well-supported separation between the two lineages, and between wild populations collected in the Eastern Cape Province in South Africa. The identification accuracy of the `ficus' and `stricta' lineages was 81.82\%. This is the first time that a method has been developed that can distinguish between these lineages. 
% The gene phylogenies created here did not show a separation between the North and South American species; contradicting previous phylogenies based on morphological characteristics. This finding ties in with current genetic evidence that the Cactaceae, and hence the Dactylopiidae as well, evolved in what is now the South American continent; and only dispersed northwards much later. \\
An additional component of this thesis was the creation of three user-friendly R-based programs to assist with:
\begin{enumerate}
    \item ISSR data processing.
    \item The identification of query \textit{Dactylopius} nucleotide sequences relative to the gene databases created here.
    \item A graphical user interface (GUI) version of the R package `SPIDER', which is useful for the assessment of the accuracy of genetic barcode data.
\end{enumerate}
A successful biological control programme relies on the correct identification of the agent in question, and so it is imperative that cactus biological control practitioners are able to distinguish between \textit{Dactylopius} species and lineages in order to release the most effective ones onto target Cactaceae.
The laboratory protocols reported, and data processing tools created here, have largely addressed this need and offer valuable practical applications. These include: 
\begin{enumerate}
    \item The flagging of potential new species, cryptic species, and lineages of dactylopiid species released as new biocontrol agents.
    \item Validating the identifications made by taxonomists based on morphology.
    \item Confirming to which species, and, where applicable, to which lineage, a field-collected sample belongs.
    \item Identifying hybrids resulting from lineage crosses.
\end{enumerate}
 Ensuring that the correct \textit{Dactylopius} species are utilised for biological control will improve the control of invasive Cactaceae and protect biodiversity and agricultural productivity.
 

