\section{Impacts of Invasive Species}
 
Alien invasive organisms have the potential to significantly alter novel environments and affect native species at the genetic through to the ecosystem level \citep{Thompson1991, Crooks, Lockwood2013}.  \citet{Vila2017} grouped the impacts of invasive species on ecosystem services into four categories; namely that of 1) supporting, 2) provisioning, 3) regulating, and 4) cultural and human well-being. \\
Supporting services include important ecosystem processes incorporating nutrient and energy flows, such as carbon sequestration and primary production, nutrient cycling, habitat structure, and hydrology. As an example, invasive \textit{Acacia mearnsii} De Wild. (black wattle) dominate riparian zones in South Africa and use significantly more water than indigenous plants. This causes major reductions in water catchment yields and streamflow \citep{Maitre2000, Dye2004}. \citet{Maitre2000} estimated that IAPs in South Africa use approximately 3.3 billion m\textsuperscript{3} of water annually. This is a very disconcerting figure taking into account that many parts of the country face severe water shortages. \\
% \textit{Spartina alterniflora} Loisel. (`smooth cordgrass'), for example, is an invader of coastal wetlands and causes a significant increase in primary production, and hence the increased availability of soil organic carbon \citep{Nie2017}. Similarly, the nitrogen-fixer \textit{Morella faya} Aiton (`fire tree') increased the nitrogen content of soils in invaded areas of Hawaii by one order of magnitude \citep{Vitousek1989}. Among other effects, this large nutrient input facilitated the secondary growth of other invasive weeds such as \textit{Psidium cattleianum} Sabine (`strawberry guava'), and altered primary succession patterns on young volcanic rock sites \citep{Vitousek1989}. A similar effect was seen in South Africa, where invasive \textit{Acacia saligna} Labill. (Wendl.) facilitated the growth of a secondary grass invader (\textit{Ehrharta calycina} Smith) in the Fynbos region due to nitrogen fixation \citep{Yelenik2018a}. Many invasive species proceed to alter habitat structure, such as altering light and water availability \citep{Yelenik2018a, Asner2008}, rates of sedimentation \citep{Daehler1996,DiTomaso1998}, erosion \citep{VanWilgen1985} and water quality \citep{VanWilgen2011,DeLange2010,Chamier2012}. For example, \textit{Eichhornia crassipes} (Solms. Laubach), one of the world’s worst aquatic weeds, alters habitats by forming dense mats over freshwater bodies \citep{Cilliers1991}. This prevents sunlight from reaching native submerged plants, contributes to evapotranspiration and sedimentation and depletes dissolved oxygen in the surrounding water \citep{Cilliers1991, VanWyk2002, VanWilgen2011}. The weed is also known to block waterways, impede water flow and damage infrastructure \citep{Cilliers1991, Catford2012}. Invaders can disrupt and modify hydrological systems \citep{LeMaitre1996, Pejchar2009, Catford2012}, such as \textit{Tamarix} L. spp. (salt cedars) in the United States \citep{DiTomaso1998} and \textit{Pinus} L. \citep{Scott1997}, \textit{Eucalyptus} L’ Hér. \citep{Dye2013}, \textit{Prosopis} \citep{Maitre2000} and \textit{Acacia} Martius \citep{VanLill1980} spp. in South Africa. \citet{Zavaleta2000}, for example, estimated that \textit{Tamarix ramosissima} Ledeb. used 1.4-3.0 billion m\textsuperscript{3} more water than native plants in the south western United States. Similarly, \textit{Acacia mearnsii} De Wild. (black wattle) dominates riparian zones and uses significantly more water than indigenous plants, which can cause major reductions in water catchment yields and streamflow \citep{Maitre2000, Dye2004}. \citet{Maitre2000} estimated that IAPs in South Africa use approximately 3.3 billion m\textsuperscript{3} of water annually. \newline 
Provisioning services include the resources humans need for food, fuel, water, and other essential materials such as fibre \citep{Pejchar2009,Vila2017}. Approximately 10\% of the Earth's land surface comprises arable land used for agriculture and horticulture \citep{Fried2017}. These areas are particularly vulnerable to biological invasions due to the high incidence of fertilisation and disturbance, where weeds can interfere through competitive, allelopathic, and parasitic mechanisms \citep{Fried2017}. \textit{Parthenium hysterophorus} L. (‘famine weed’), for example, has invaded agricultural lands in parts of Africa, Asia, and Australia \citep{Mcconnachie2011}. The weed is highly competitive and allelopathic, and releases phenolics and sesquiterpenes into the surrounding soil that inhibits the growth of important crops such as cereals, vegetables, and grasses \citep{Evans1997}. \\
% \textit{Centaurea solstitialis} L. (yellow star thistle), for example, is a weed native to Europe that is unpalatable to grazing livestock, and causes major losses in forage yields in California \citep{Eagle2007}. Another highly damaging weed native to Central America, \textit{Parthenium hysterophorus} L. (‘famine weed’), has invaded agricultural lands in parts of Africa, Asia and Australia \citep{Mcconnachie2011}. The weed is highly competitive and allelopathic, releasing phenolics and sesquiterpenes into the surrounding soil that inhibits the growth of important crops such as cereals, vegetables and grasses \citep{Evans1997}. Some recorded crop losses are severe, such as an approximate 40-97\% loss in sorghum yield in Ethiopia \citep{Tamado2002InterferenceCompetition}, crop losses of up to 40\% in India \citep{Kohsla1981} and a reduction in the carrying capacity of some farms in Australia and India by approximately 40 and 90\%, respectively \citep{McFadyen1992, Evans1997}. Additionally, the weed can serve as a reservoir for crop pests \citep{Robertson}, degrade grazing areas \citep{McFadyen1992} and cause severe health implications for livestock (such as gastrointestinal irritation, dermatitis, anorexia and death) \citep{Evans1997}. Many invasive aquatic macrophytes cause large-scale damage to fisheries and nursery grounds, such as \textit{Lagarosiphon major} Ridl. (`curly water weed'), \textit{Myriophyllum aquaticum} Vell. Verd. (`parrot’s-feather'), \textit{Crassula helmsii} Kirk. Cockayne (`New Zealand pigmyweed'), \textit{Alternanthera philoxeroides} Griseb (`alligator weed'), \textit{Lythrum salicaria} L. (`purple loosestrife') and \textit{Caulerpa taxifolia} M. Wahl. (`killer alga') \citep{Gozlan2017}. These weeds tend to deplete dissolved oxygen, alter water chemical profiles, decrease light penetration, outcompete native plant life and ultimately modify community structures and trophic interactions \citep{Schultz2012, Gozlan2017}. 
Regulating services are those that sustain and adjust an ecosystem, such as pollination services, disturbance regimes, food webs, water purification, soil stabilization, climate regulation, and the role of endemic species (e.g. mutualistic relationships) \citep{Pejchar2009, Vila2017}. As an example of how a disturbance regime can be affected, invasive plants can alter fire regimes by serving as a fuel source and promoting more intense fires \citep{DAntonio1992, Brooks2004,Milton2004,LeMaitre2014}. This allows invaders to outcompete native species that are not suitably adapted to this disturbance and do not recover as quickly \citep{Milton2004}. \\
% Invasive plants can alter fire regimes by serving as a fuel source and promoting frequent fires \citep{DAntonio1992, Brooks2004,Milton2004,LeMaitre2014}. This allows invaders to outcompete native species that are not suitably adapted to this disturbance and do not recover as quickly \citep{Milton2004}. Invasive perennial grasses are particularly notorious \citep{DAntonio1992}, such as \textit{Arundo donax} L. (`Spanish reed'), \textit{Nassella tenuissima} Trin. Barkworth (`tussock grass') and \textit{Pennisetum setaceum} Forssk. Chiov. (‘crimson fountain grass’) \citep{Milton2004}. \citet{VanWilgen1985} conducted a fire simulation study using \textit{Hakea sericea} Schrad. (`silky hakea') and \textit{Acacia saligna} (Labill.) Wendl., (`Port Jackson willow') and found that these two shrubs increased fuel load by 60\% and 50\%, respectively. Fires also alter nutrient budgets in ecosystems by volatizing and converting elements such as nitrogen and carbon into more bioavailable forms \citep{DAntonio1992}. The impacts of invasive species on food webs are extensive (see a review by \citet{David2017a}). One classic example is that of the invasive \textit{Anoplolepsis gracilipes} Smith (Hymenoptera: Formicidae) (`yellow crazy ant') on Christmas Island \citep{ODowd2003}. \textit{Gecarcoidea natalis} Pocock (`red land crab') is endemic, and serves as a keystone species in ecosystem functioning in the forests on the island \citep{ODowd2003}. The crustacean regulates processes such as seedling recruitment, leaf litter degradation and the density of litter invertebrates \citep{Green1997, Green1999a}. \textit{Anoplolepsis gracilipes} form supercolonies, and attack and kill \textit{G. natalis} by spraying formic acid into their eyes and mouths \citep{ODowd2003}. The crabs are particularly vulnerable to attack during their annual migration through the forest to the ocean shore \citep{ODowd2003}. Declines in the population numbers of \textit{G. natalis} have been shown to have both bottom-up and top-down effects on the forest ecosystem that affect multiple trophic levels; namely through the change in vegetation composition and the association of scale insects with the invasive ant \citep{ODowd2003}. These sap-sucking scale insects cause die-back of the forest canopy cover \citep{ODowd2003}. \newline 
Cultural and human well-being services entail the aesthetics of nature and the spread of diseases \citep{Vila2017}. For example, \textit{Eichhornia crassipes} infestations in fresh water bodies limits boating activity and other water recreation activities \citep{Villamagna2018}. Invaded areas tend to display lower diversity, where invasive species tend to dominate the landscape and reduce its aesthetic value \citep{Kueffer2017}. Invasive species can act as vectors and reservoirs of diseases, such as \textit{Aedes aegypti} L. (`yellow fever mosquito') and \textit{A. albopictus} Skuse (`Asian tiger mosquito') (Diptera: Culicidae) that vector the Chikungunya virus, and \textit{Phlebotomus} Loew and \textit{Lutzomyia} França spp. (`sandflies') (Diptera: Psychodidae) that vector Leishmania pathogens \citep{Rabitsch2017}. \\
% Conflicts of interest arise when an invasive species holds economic value in its introduced range, such as \textit{Opuntia ficus-indica} L. (Mill.), which is both highly invasive and a source of income for many rural communities in South Africa \citep{Brutsch1993ThePlants, Beinart2011}. The prickly pear fruit is harvested and sold, and is also used to make beer, jam, and traditional medicines \citep{Brutsch1993ThePlants, shackleton2011invasive}. Rural communities in the Eastern Cape in particular rely heavily on the fruit for an income, and the local people are generally unaware of its invasive status   \citep{shackleton2007assessing}. Greater communication between these informal traders and the municipality is necessary, where legislation needs to be implemented that achieves a balance between limiting the negative effects of the invader on the environment, and promoting economic growth \citep{shackleton2011invasive}. \\
% Cultural services associated with the recreation and tourism industry are relatively easier to quantify, such as the cost associated with \textit{Myriophyllum spicatum} L. (`Eurasian water milfoil') on Lake Tahoe in the USA, and the decline in outdoor activities due to the lacerations caused by \textit{Centaurea solstitialis} L. (`yellow star thistle') and stings by \textit{Solenopsis invicta} Buren (`red imported fire ant') \citep{Pejchar2009}. Other invasive species, such as venomous spiders (e.g. \textit{Latrodectus hasselti} Thorell, `widow spiders' (Araneae: Theridiidae)), wasps (e.g. \textit{Vespula germanica} Fabricius, `German yellowjacket' (Hymenoptera: Vespidae)), fish (e.g. \textit{Pterois volitans} L., `lionfish' and \textit{Lagocephalus sceleratus} Gmelin, `pufferfish'), snakes (e.g. \textit{Boiga irregularis} Merrem, `brown tree snake') and plants (e.g. \textit{Solanum nigrum} L., `European black nightshade'), can be harmful to humans, sometimes even fatal \citep{Nentwig2017}. 
The impacts associated with invasive Cactaceae touch on all four invasion impact categories, predominantly relating to the loss of biodiversity, ecological functioning, and agricultural output \citep{kaplan2017proposed, novoa2019spinelessness}. Areas of dense cactus infestation leads to the exclusion of indigenous plants and animals through competitive exclusion and damage caused by spines and glochids \citep{walters2011}. Negative impacts on agricultural practices are caused by reduced grazing areas for livestock and the resulting decrease in productivity, costs incurred by mechanical and chemical control practices (where the costs of control sometimes exceed the value of the land itself), injuries to livestock caused by spines, and damage to fleece and hides \citep{Lloyd2014}. 